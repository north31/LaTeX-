\documentclass{article} 

\usepackage{ctex} 
\usepackage{amsmath}

\newcommand\degree{^\circ} %采newcommand声明degree,^\circ 显示的是右上标的圈,即°
\newtheorem{theorem}{定理}[section]
\newtheorem{definition}{定义}[section]
%\renewcommand{\thefootnote}{\roman{footnote}} %i, ii, iii 

\title{\heiti 我的LaTeX练习01} 
\author{\kaishu 董秦 2016210123} 
\date{\today} 
\begin{document} 
	\maketitle
	\begin{abstract}
		文章为练习使用,是本人第一次用LaTeX写的,LaTeX还是需要多加练习才能掌握,希望之后可以熟练使用。
	\end{abstract}
    \tableofcontents
    \newpage
    \section{数学}
    \subsection{勾股定理} %\footnote{Pythagoras,约公元前580年—约前500(490)年}
    \subsubsection{语言描述}
    \begin{theorem}
    	{勾股定理}:直角三角形斜边的平方等于两腰的平方和。
    \end{theorem}
    \subsubsection{数学描述}
    设直角三角形$ABC$,其中$\angle C=90\degree$, 则有:
    \begin{equation} 
    AC^2 = AB^2 + BC^2
    \end{equation}
    
    \subsection{泰勒公式} %\footnote{Brook Taylor,1685年8月18日-1731年11月30日}
    \subsubsection{语言描述}
    \begin{definition}
    	{泰勒公式}: 可以用若干项连加式来表示一个函数,这些相加的项由函数在某一点的导数求得。
    \end{definition}
    \subsubsection{数学描述}
    \begin{gather}
    f(x) = \frac{f(a)}{0!}+\frac{f(a)^\prime}{1!}(x-a)+\frac{f\prime\prime(a)}{2!}(x-a)^2+\cdots+\frac{f^n (a)}{n!}(x-a)^n+R_n(x)
    \end{gather}
    
    \section{物理}
    \subsection{质能方程} %\footnote{Albert Einstein,1879年3月14日—1955年4月18日}
    \subsubsection{质能方程}
    \begin{definition}	
        {质能方程}即描述质量与能量之间的当量关系的方程。
    \end{definition}
    \subsubsection{数学描述}  
     \begin{equation} 
        E=mc^2 
     \end{equation}
\end{document}
